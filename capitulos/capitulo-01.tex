\chapter{Introdução}\label{cap:introducao}
A compressão de dados é essencial para minimizar custos de armazenamento e
transmissão, reduzindo o volume de dados sem comprometer o entendimento da
informação. No caso da compressão sem perdas, segundo a teoria de Shannon, a
entropia da fonte estabelece o limite inferior para a taxa média de bits de
qualquer esquema de compressão~\cite{mackay2003information}. Para aproximar-se
desse limite teórico, algoritmos de compressão combinam técnicas de dicionário
e codificação estatísticas.

Na prática, formatos de compressão amplamente utilizados empregam essa
abordagem híbrida. Por exemplo, o algoritmo DEFLATE, utilizado no formato ZIP,
aplica LZ77 em conjunto com a codificação de Huffman para obter compressões
melhores. Essa combinação explora tanto os padrões repetitivos de longo alcance
quanto as probabilidades de ocorrência dos símbolos, elevando a eficiência
global do método \cite{rfc1951}.

De modo geral, os algoritmos de compressão sem perdas dividem-se em métodos de
codificação estatística e métodos de dicionário \cite{sayood2012introduction}.
Por exemplo, a codificação de Huffman e a codificação aritmética são técnicas
estatísticas, a ultima capaz de representar toda a mensagem como um único
número fracionário no intervalo $[0, 1[$, alcançando compressões muito próximas
do limite de entropia. Por sua vez, os métodos de dicionário exploram
redundâncias substituindo sequências repetitivas por referências a ocorrências
anteriores já vistas: um dos mais conhecidos é o algoritmo LZ77, proposto por
Ziv e Lempel\cite{1055714}, que implementa esse princípio construindo
dinamicamente um ``dicionário'' de padrões enquanto lê os dados.

Apesar do uso disseminado dessas técnicas em aplicações, a biblioteca de código
aberto Komm não dispunha até o momento de uma implementação do LZ77. Essa biblioteca voltada ao ensino e à simulação em
comunicação digital, já inclui diversos algoritmos clássicos de compressão,
como os código de Huffman, Shannon--Fano e LZ78, evidenciando a relevância de
incorporar as técnicas LZ77 para torná-la mais
completa. Portanto, este trabalho tem como objetivo desenvolver e integrar um
versão do LZ77 na biblioteca Komm, avaliando estes
algoritmos em termos de taxa de compressão, tempo de processamento e uso de
memória.

\section{Objetivo geral}
Expandir as capacidades da biblioteca Komm com a implementação dos algoritmos
LZ77.

\section{Objetivos Especificos}
\begin{itemize}
    \item  Projetar e desenvolver módulos da codificação LZ77 na arquitetura da Komm,
          observando padrões de codificação e cobertura de testes e escrita da
          documentação.

    \item Validar a compressão e descompressão em arquivos de textos e imagens,
          assegurando corretude e integridade.

    \item Realizar análise comparativa de desempenho (taxa de compressão, tempo de
          execução e uso de memória) em relação a Huffman, LZ78 e LZW.

\end{itemize}
\section{Organização do Texto}
O resto deste trabalho está organizado da seguinte forma: o Capítulo
\ref{cap:fundamentacao} apresenta os um estudo dos conceitos teóricos da
compressão sem perdas LZ77. O Capítulo \ref{cap:desenvolvimento} detalha a
implementação do algoritmo na biblioteca Komm. O Capítulo
\ref{cap:resultados} discute os resultados experimentais obtidos. Finalmente,
o Capítulo \ref{cap:conclusao} conclui o trabalho e sugere direções para
trabalhos futuros.