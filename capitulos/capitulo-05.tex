\chapter{Conclusão e Trabalhos Futuros}\label{cap:conclusao}

Este trabalho teve como objetivo implementar o algoritmo de compressão sem perdas
\textit{Lempel–Ziv 77} (LZ77) na biblioteca \textit{Komm}, integrando-o ao conjunto de
códigos-fonte abertos já disponíveis para o estudo de sistemas de comunicação digitais.
A implementação buscou seguir fielmente o modelo descrito no artigo original de
Ziv e Lempel~\cite{1055714}, priorizando clareza didática, compatibilidade com a
infraestrutura existente da biblioteca e modularidade para futuras extensões.

O desenvolvimento foi dividido em quatro módulos internos de codificação e decodificação,
permitindo a inspeção direta dos tokens gerados e das etapas intermediárias do processo.
Essa abordagem reforçou o caráter educacional da biblioteca, facilitando a visualização
do funcionamento interno do LZ77 e a comparação com outros algoritmos já implementados,
como Huffman, LZW e LZ78.

Os resultados experimentais confirmaram o comportamento clássico do LZ77:
aumentos no tamanho da janela deslizante (\(W\)) proporcionaram melhor taxa de compressão,
mas com crescimento proporcional do tempo de execução e do consumo de memória.
A comparação com implementações externas, em especial o \textit{FastLZ} em C,
demonstrou que o desempenho inferior da versão em Python deve-se principalmente
à ausência de estruturas de indexação otimizadas (como tabelas de hash) e ao
modelo de tokens de tamanho fixo, que privilegia a simplicidade conceitual
em detrimento da eficiência máxima.

Apesar dessas limitações, a integração do LZ77 à \textit{Komm} mostrou-se bem-sucedida:
todas as simulações preservaram a integridade dos dados, e a estrutura modular
proposta oferece uma base sólida para pesquisa, ensino e experimentação de novos
métodos de compressão.

\section{Trabalhos Futuros}

Como desdobramento natural deste projeto, destacam-se as seguintes possibilidades
de evolução:

\begin{itemize}
  \item \textbf{Implementação da Codificação Aritmética:} continuação da proposta
  original, incorporando o algoritmo de codificação aritmética à \textit{Komm},
  o que permitirá avaliar esquemas híbridos e comparações diretas com Huffman
  e Tunstall em termos de entropia e eficiência.

  \item \textbf{Otimização do LZ77 com Tabelas de Hash:}
  desenvolver uma versão alternativa do \textit{LempelZiv77Code}
  que utilize uma tabela de dispersão (\textit{hash table}) para acelerar
  a busca de correspondências, aproximando-se do desempenho de implementações
  como o \textit{FastLZ}, sem perder a clareza e a compatibilidade do modelo atual.

  \item \textbf{Análise de Compressão Mista:}
  investigar combinações entre LZ77 e codificações de entropia (Huffman ou Aritmética),
  como ocorre no formato \textit{DEFLATE}, avaliando o impacto na taxa e no tempo
  de compressão.

\end{itemize}
