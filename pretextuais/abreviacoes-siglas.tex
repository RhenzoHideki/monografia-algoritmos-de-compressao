\makeglossaries
% \setabbreviationstyle[acronym]{long-postshort-user}
\setabbreviationstyle[acronym]{long-short-user}
\glssetcategoryattribute{acronym}{nohyperfirst}{true}
% \glssetcategoryattribute{acronym}{nohyper}{true}

% ------------------------------------%
%  Como usar os comandos do glossário
% ------------------------------------%
% \gls{TLS} - Na 1a. vez, texto por extenso + sigla. Nas chamadas subsequentes somente a sigla
% \glspl{IdP} - Para colocar s como prefixo da sigla
% \Gls{} - Inicial em maiúscula
% \glsxtrfull{} - texto por extenso + sigla
% \glsxtrfullpl{} - texto por extenso no plural + sigla
% \glsxtrshort{} - somente a sigla
% \glsxtrlong{} - somente o texto por extenso

%  Caso queira criar uma entrada que tenha tradução para inglês, use o parâmetro user1. 
%      Veja exemplo abaixo para Credenciais Verificáveis
% 
% Caso queira criar plural, use o parâmetro longplural. Veja exemplo para IAA
% ------------------------------------%

\newacronym
{json} % rótulo 
{JSON} % sigla
{\textit{JavaScript Object Notation}} % por extenso


\newacronym{ABNT}{ABNT}{Associação Brasileira de Normas Técnicas}

\newacronym{abnTeX}{abnTeX}{ABsurdas Normas para TeX}

\newacronym[longplural={Autoridades Certificadoras}]{AC}{AC}{Autoridade Certificadora}

\newacronym{AES}{AES}{\textit{Advanced Encryption Standard}}

\newacronym{TLS}{TLS}{\textit{Transport Layer Security}}

\newacronym{TPC}{TPC}{Terceira Parte Confiável}

\newacronym{IFSC}{IFSC}{Instituto Federal de Santa Catarina}